\documentclass{beamer}

\usetheme{Warsaw}
\usecolortheme{seahorse}


\usepackage[french]{babel}
\usepackage[utf8]{inputenc}
\usepackage[T1]{fontenc}

\AtBeginSection[]
{
    \begin{frame}
        \frametitle{Table des matières}
        \tableofcontents[currentsection]
    \end{frame}
}


\begin{document}

\title{Algorithmes approchés}

\author{Arpad Rimmel}
\institute{SUPELEC}


\begin{frame}
    \titlepage
\end{frame}


\section{Introduction}


\begin{frame}
    \frametitle{Introduction}
    
    Catégories de problèmes étudiés:
    \begin{itemize}
        \item Problèmes où l'espace de recherche est trop vaste pour être exploré en entier.

        Exemple: exploration d'arbre avec un grand nombre de nœuds.

        \item Problèmes où il est impossible de garantir une solution optimale.

        Exemple: optimisation d'une fonction inconnue.
    \end{itemize}

    $\Rightarrow$ Utilisation d'algorithmes donnant la meilleur solution possible en un temps donné.

\end{frame}


\section{Travaux précédents}

\begin{frame}
    \frametitle{Exploration d'arbre}
    Problème: prendre des décisions dans un environnement discret, observable, avec horizon fini et avec récompenses.
\\~
    \begin{columns}
    \begin{column}[l]{5cm}
        Exemple:
        \begin{itemize}
            \item multiplication de matrices
            \item jeu de Go
            \item samegame
            \item POMDP
            \item ...
        \end{itemize}
    \end{column}
    \begin{column}[r]{5cm}
        \begin{center}
            \includegraphics[scale=0.4]{tree.jpg}
        \end{center}
    \end{column}
    \end{columns}

\end{frame}

\begin{frame}
    \frametitle{Bandit Based Monte Carlo Tree Search}
    \begin{itemize}
        \item Basé sur la formule du bandit
        \item Évaluation par simulation Monte Carlo
        \item Performant pour les applications où la première décision est importante
    \end{itemize}
    \begin{center}
        \includegraphics[scale=0.5]{mcts-algorithm.png}
    \end{center}

\end{frame}

\begin{frame}
    \frametitle{Nested Monte Carlo Tree Search}
    \begin{itemize}
        \item Basé sur une récursion d'évaluations
        \item Évaluation par simulation Monte Carlo
        \item Performant pour les applications où toutes les décisions sont importantes
    \end{itemize}
    \begin{center}
        \includegraphics[scale=0.5]{nested-algorithm.jpg}
    \end{center}

\end{frame}


\begin{frame}
    \frametitle{Optimisation}
    Problème: 
    \begin{itemize}
    \item Trouver le maximum d'une fonction "boite noire".
    \item on peut obtenir la valeur en un point donné.
    \item Espace de recherche trop grand pour être exploré en entier.
    \end{itemize}

    \begin{center}
        \includegraphics[scale=0.5]{multimodalFitnessLandscape.jpg}
    \end{center}

\end{frame}

\begin{frame}
    \frametitle{Algorithmes Evolutionnaires}

    \begin{itemize}
        \item Basé sur l'évolution d'une population d'individus
        \item Chaque individu correspond à un point 
        \item Le score de chaque individu correspond à la valeur de la fonction en ce point
        \item De nouveaux individus sont générés à partir des scores précédents
    \end{itemize}
    \begin{center}
        \includegraphics[scale=0.3]{EAalgorithmPedagogical.png}
    \end{center}


\end{frame}




\section{Travaux actuels}

\begin{frame}
    \frametitle{Carrés latins}
    Applications:
    \begin{itemize}
        \item Planification d'expériences
    \end{itemize}
    But:
    \begin{itemize}
        \item Placer $n$ points dans une hypercube de taille $n$ et de dimension $d$.
        \item \textit{non-collapsing}: pour une dimension donnée, chaque point doit avoir une valeur différente.
        \item \textit{space-filling}: trouver la solution où la distance minimale entre 2 points est maximale.
    \end{itemize}
    \begin{center}
        \includegraphics[scale=0.3]{LHS.jpg}
    \end{center}

\end{frame}

\begin{frame}
    \frametitle{Etat de l'art}
    \begin{itemize}
    \item En dimension 2, pour les normes $L_1$ et $L_{inf}$, des algorithmes donnant la solution optimale en un temps linéaire ont été fournis.
    \item Pour la norme $L_2$, uniquement des solutions approchées.
    \item En dimension supérieure à 2, uniquement des solutions approchées.
    \item Un site internet regroupe les meilleurs solutions existantes pour un grand nombre de dimensions et de tailles.
    \end{itemize}
\end{frame}

\begin{frame}
    \frametitle{Travaux préliminaires}
    Pour le moment 2 algorithmes testés:
    \begin{itemize}
        \item Nested Monte Carlo Tree Search

        Problème: pas vraiment une structure d'arbre.
        \item Algorithme génétique

        Résultat prometteurs.
    \end{itemize}
\end{frame}


\section{Perspectives}

\begin{frame}
    \frametitle{Perspectives}
    Carrés latins:
    \begin{itemize}
        \item Améliorations des algorithmes existants (fonction d'évaluation, solution par patterns)
        \item Nouveaux algorithmes
        \item Solutions exactes
    \end{itemize}
    Autres thèmes de recherche:
    \begin{itemize}
        \item trouver des chemins dans des graphes 
        
        Application aux problèmes de congestion en télécommunication
        \item trouver des colorations pondérées dans des graphes 

        Application aux problèmes de répartition de fréquences pour les utilisateurs d'un réseau
    \end{itemize}
\end{frame}



\end{document}
